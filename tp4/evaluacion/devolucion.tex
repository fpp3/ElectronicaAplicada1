\documentclass[]{informeutn}

% Datos del informe
\materia{Electrónica Aplicada I}
\titulo{Devolución Expositores CC}
\comision{3R2}
\autores{
          Gaston Grasso, 401892\\
          Franco Palombo, 401910\\
          Angelo Prieto, 401012}
\fecha{17/09/2025}

\begin{document}
  \maketitle

    \chapter{Grupo: Coronel, Garcia, Olmedo, Castilla}
      \section{Fortalezas}
        \begin{itemize}
            \item Buena animación en tiempo real de las rectas de carga, excursiones y fases.
        \end{itemize}
      \section{Sugerencias}
        \begin{itemize}
            \item Mejorar la profundidad en la explicación de los cálculos y mediciones de pequeña señal. No resulto
              clara la explicación presentada. Falto teoría.
            \item Mejorar la presentación de los datos calculados y simulados. Cuando son valores continuos (punto Q),
              conviene realizar simulaciones de "Operating Point" (.op) para obtener los valores exactos. Estos pueden
              ser comparados con mucha mas facilidad en una tabla.
        \end{itemize}
      \section{Rubrica de evaluación}
        \begin{table}[!ht]
          \centering
          \begin{tabular}{|p{4.5cm}|p{7.5cm}|c|}
            \hline
            \textbf{Criterio} & \textbf{Descripción} & \textbf{Calificación} \\ \hline
            \textbf{Claridad expositiva} & ¿Se explicaron los conceptos de manera comprensible y ordenada? & \textbf{3} \\ \hline
            \textbf{Rigor Técnico} & ¿Los cálculos, simulaciones y conclusiones son correctos y bien fundamentados? & \textbf{3.6} \\ \hline
            \textbf{Objetivos Completados} & ¿Se presentaron resultados para todos los puntos solicitados en la consigna? & \textbf{5} \\ \hline
            \textbf{Respuestas Preguntas} & ¿Respondieron con precisión y profundidad a las preguntas del público? & \textbf{4.8} \\ \hline
          \end{tabular}
        \end{table}

    \chapter{Grupo: Bonelli, Rios, Watson}
      \section{Fortalezas}
        \begin{itemize}
            \item Presentación muy dinámica y llevadera.
            \item Buena profundidad teórica. No fue ni mucho ni poco.
        \end{itemize}
      \section{Sugerencias}
        \begin{itemize}
            \item Hablarle mas a la clase y no tanto al profe o al pizarron.
        \end{itemize}
      \section{Rubrica de evaluación}
        \begin{table}[!ht]
          \centering
          \begin{tabular}{|p{4.5cm}|p{7.5cm}|c|}
            \hline
            \textbf{Criterio} & \textbf{Descripción} & \textbf{Calificación} \\ \hline
            \textbf{Claridad expositiva} & ¿Se explicaron los conceptos de manera comprensible y ordenada? & \textbf{3.6} \\ \hline
            \textbf{Rigor Técnico} & ¿Los cálculos, simulaciones y conclusiones son correctos y bien fundamentados? & \textbf{4.6} \\ \hline
            \textbf{Objetivos Completados} & ¿Se presentaron resultados para todos los puntos solicitados en la consigna? & \textbf{5} \\ \hline
            \textbf{Respuestas Preguntas} & ¿Respondieron con precisión y profundidad a las preguntas del público? & \textbf{4.6} \\ \hline
          \end{tabular}
        \end{table}

    \chapter{Grupo: Lopez, Reinoso, Sibona, Suppo}
      \section{Fortalezas}
        \begin{itemize}
            \item Presentación muy dinámica y llevadera.
        \end{itemize}
      \section{Sugerencias}
        \begin{itemize}
            \item Podría tener un poco mas de profundidad teórica.
        \end{itemize}
      \section{Rubrica de evaluación}
        \begin{table}[!ht]
          \centering
          \begin{tabular}{|p{4.5cm}|p{7.5cm}|c|}
            \hline
            \textbf{Criterio} & \textbf{Descripción} & \textbf{Calificación} \\ \hline
            \textbf{Claridad expositiva} & ¿Se explicaron los conceptos de manera comprensible y ordenada? & \textbf{4.6} \\ \hline
            \textbf{Rigor Técnico} & ¿Los cálculos, simulaciones y conclusiones son correctos y bien fundamentados? & \textbf{4.6} \\ \hline
            \textbf{Objetivos Completados} & ¿Se presentaron resultados para todos los puntos solicitados en la consigna? & \textbf{5} \\ \hline
            \textbf{Respuestas Preguntas} & ¿Respondieron con precisión y profundidad a las preguntas del público? & \textbf{5} \\ \hline
          \end{tabular}
        \end{table}

\end{document}
