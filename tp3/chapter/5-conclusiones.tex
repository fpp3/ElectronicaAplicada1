\chapter{Conclusiones}
El amplificador en emisor común quedó correctamente fijado en MES. El punto de reposo medido fue \(I_{CQ}\approx 6.51\,
\mathrm{mA}\) y \(V_{CEQ}\approx 3.17\,\mathrm{V}\), valores cercanos a lo previsto y dentro del margen razonable de
\(\pm10\%\). No se observaron indicios de recorte en el entorno de señal utilizado.

En pequeña señal, la impedancia de salida resultó \(Z_o\approx 1.18\,\mathrm{k}\Omega\), coherente con \(R_C\approx
1.2\,\mathrm{k}\Omega\). La impedancia de entrada fue mayor que la analítica (\(\sim 3.18\,\mathrm{k}\Omega\) medida
frente a \(1.53\,\mathrm{k}\Omega\) calculada), lo que sugiere dispersión de \(\beta\) y efecto de la red de
polarización sobre \(h_{ie}\). La ganancia de tensión en magnitud fue menor que el modelo (\(|A_v|\approx 76.9\) vs
\(141.8\)), mientras que la ganancia de corriente se mantuvo en el orden esperado (\(|A_i|\approx 244\)). La inversión
de fase propia del emisor común no aparece en las capturas por trabajarse con valores pico.

Como lectura integradora, los datos confirman que el modelo híbrido describe bien tendencias y órdenes de magnitud 
cuando el circuito está en MES: \(Z_o\) queda dominada por \(R_C\) y \(Z_i\) 
depende fuertemente de \(h_{ie}\). En este marco, pequeñas variaciones de parámetros (tolerancias y dispersión de 
dispositivo) modifican \(|A_v|\) y \(Z_i\) en un rango acotado sin desplazar el punto Q de la zona lineal. En síntesis, 
el montaje cumple su objetivo: ofrece ganancia estable, comportamiento reproducible y una relación 
clara entre diseño teórico y respuesta medida.
