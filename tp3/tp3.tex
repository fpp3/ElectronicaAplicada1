\documentclass[chaptersright]{informeutn}
\usepackage{circuitikz}

% Datos del informe
\materia{Electrónica Aplicada I}
\titulo{Trabajo Práctico 3}
\comision{3R2}
\autores{
          Gaston Grasso - 401892\\
          Franco Palombo - 401910\\
          Angelo Prieto - 401012}
\fecha{20/08/2025}

\begin{document}
  \maketitle
  \tableofcontents
  \setcounter{page}{1}
  \thispagestyle{plain}

\chapter{Introducción}

\chapter{Diseño para Máxima Excursión Simétrica}
\begin{figure}[H]
    \centering
    \begin{tikzpicture}
    	\node[npn](N1) at (9.25, 5.75){} node[anchor=west] at (N1.text){$Q1$};
    	\draw (7, 5) to[american resistor, l={$R_1$}] (7, 2.75);
    	\draw (7, 5.75) to[capacitor, l={$C_1$}] (3, 5.75);
    	\draw (7, 5.75) -- (8.41, 5.75);
    	\draw (9.25, 5) to[american resistor, l={$R_E$}] (9.25, 2.75);
    	\draw (3, 5.75) to[sinusoidal voltage source, l={$v_i$}] (3, 2.5);
    	\draw (7, 8.75) to[american resistor, l={$R_2$}, name=R1] (7, 6.5);
    	\draw (9.25, 8.77) to[american resistor, l={$R_c$}] (9.25, 6.52);
    	\draw (7, 5) -| (7, 6.5);
    	\draw (11, 5) to[capacitor, l={$C_E$}] (11, 2.75);
    	\draw (9.25, 5) -- (11, 5);
    	\draw (7, 8.75) -| (7, 9) -| (9.25, 8.77);
    	\draw (11, 2.75) |- (7, 2.5) -| (7, 2.75);
    	\draw (9.25, 2.75) -| (9.25, 2.5);
    	\draw (3, 2.5) -- (7, 2.5);
    	\draw (10.25, 6.5) to[capacitor, l={$C_2$}] (13, 6.5);
    	\draw (13.75, 5.5) to[american resistor, l={$R_L$}] (13.75, 3.25);
    	\draw (13.75, 6.5) -| (13.75, 5.5);
    	\draw (11, 2.5) -| (13.75, 3.25);
    	\node[vcc](N2) at (9.25, 9.75){} node[anchor=south] at (N2.text){$V_{cc}$};
    	\draw (9.25, 9) -| (9.25, 9.75);
    	\node[ground] at (9.25, 2){};
    	\draw (9.25, 2.5) -| (9.25, 2);
    	\draw (10.25, 6.5) -- (9.25, 6.5);
    	\draw (13, 6.5) -- (13.75, 6.5);
    \end{tikzpicture}
    \caption{Amplificador emisor común}
    \label{fig:amplificador}
\end{figure}

\section{Enunciado y datos iniciales}
    % Copiar aquí el planteo y los datos (RE, RC, RL, elección de transistor y VCC)
  \section{Cálculo de R1 y R2}
    \subsection{Teorema de Thévenin en red de entrada}
    \subsection{Análisis de malla de entrada}
    \subsection{Criterios de diseño y fórmulas generales}
    \subsection{Cálculo final de R1 y R2}
  \section{Simulación}
    \subsection{Simulación con valores de diseño}
    \subsection{Simulación con valores normalizados}
    \subsection{Comparación con cálculos analíticos}
  \section{Implementación y mediciones}
    \subsection{Mediciones en circuito implementado}
    \subsection{Consideraciones de medición}
    \subsection{Cálculo de parámetros a partir de mediciones}

\chapter{Análisis y trazado de rectas de carga}
  \section{Recta de carga en corriente continua (CC)}
    \subsection{Ecuación y puntos de corte}
  \section{Recta de carga en corriente alterna (CA)}
    \subsection{Ecuación y puntos de corte}
  \section{Obtención experimental de parámetros}
    \subsection{Corrientes en el divisor resistivo}
    \subsection{Verificación de resultados}

\chapter{Mediciones de pequeña señal}
  \section{Análisis teórico (método analítico)}
    \subsection{Circuito híbrido equivalente}
    \subsection{Cálculo de $Z_i$, $Z_o$, $A_i$ y $A_v$}
  \section{Análisis experimental}
    \subsection{Impedancia de entrada}
    \subsection{Ganancia de tensión}
    \subsection{Ganancia de corriente}
    \subsection{Impedancia de salida}


\end{document}
