\documentclass[chaptersright]{informeutn}

% Datos del informe
\materia{Electronica Aplicada}
\titulo{Trabajo Práctico 1}
\comision{3R2}
\autores{
          Gaston Grasso - 401892\\
          Franco Palombo - 401910\\
          Angelo Prieto - 401012}
\fecha{1-1-1}

\begin{document}
  \maketitle

  \tableofcontents
  \setcounter{page}{1}
  \thispagestyle{plain}git@github.com:fpp3/ElectronicaAplicada1.git

  \chapter{Introducción} 

    En el marco de la asignatura \textit{Electrónica Aplicada I}, se llevó a cabo un trabajo práctico centrado en la
    construcción y análisis de una fuente de alimentación de tensión variable. Este tipo de fuente resulta fundamental
    en entornos de laboratorio y desarrollo electrónico, ya que permite alimentar circuitos con distintos
    requerimientos de tensión de forma estable y segura.

    El diseño de la fuente fue provisto por el docente a cargo, por lo que el enfoque del trabajo no estuvo en la etapa
    de diseño conceptual, sino en la interpretación, armado y validación funcional del circuito. La actividad comenzó
    con una clase teórica introductoria, donde se abordaron los fundamentos técnicos del proyecto y se detallaron los
    criterios de análisis a aplicar en laboratorio. Al finalizar la clase, el docente realizó la entraga de los 
    materiales necesarios para avanzar con la construcción.

    La realización del proyecto se desarrolló de manera progresiva, abordando dos etapas principales de la fuente
    transformación, rectificación, filtrado primero, y luego regulacióno. Esta metodología permitió realizar ensayos
    parciales, facilitando la comprensión del comportamiento eléctrico en cada etapa.

    El circuito final tiene la capacidad de entregar una tensión de salida variable entre 0 y 30\,V con una corriente
    máxima de 1.5\,A. Se efectuaron ensayos destinados a verificar su desempeño, incluyendo mediciones de
    \textit{ripple}, control de regulación de voltaje, y análisis de temperatura en el regulador LM317. Más allá de su
    valor académico, la fuente construida será utilizada como herramienta de alimentación en futuros trabajos
    prácticos, reafirmando así su utilidad práctica y formativa.
     
\end{document}
