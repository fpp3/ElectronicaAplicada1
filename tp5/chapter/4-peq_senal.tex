\chapter{Mediciones de pequeña señal}
  \section{Análisis teórico (método analítico)}
   
    \subsection{Circuito híbrido equivalente}
    \begin{tikzpicture}
	  % Paths, nodes and wires:
    	\draw (4, 3) to[sinusoidal voltage source, l={$I_i$}] (4, 7);
    	\draw (6, 3) to[american resistor, l={$R_2$}] (6, 7);
    	\draw (8, 3) to[american resistor, l={$R_1$}] (8, 7);
    	\draw (10, 3) to[american resistor, l_={$h_{ie}$}] (10, 7);
    	\draw (14.25, 7) to[american current source, l_={$i_b . h_{fe}$}] (14.25, 3);
    	\draw (16, 3) to[american resistor, l_={$R_C$}] (16, 7);
    	\draw (18, 3) to[american resistor, l_={$R_L$}] (18, 7);
    	\node[ground] at (12, 3){};
    	\node[shape=rectangle, draw={rgb,255:red,139;green,19;blue,19}, line width=1pt, minimum width=5.465cm, minimum height=2.215cm] at (12.25, 5.125){};
    	\draw (6, 7) -- (8, 7);
    	\draw (10, 7) -- (8, 7);
    	\draw (5.75, 7) -- (6, 7);
    	\draw (4, 7) -- (5.75, 7);
    	\draw (4, 3) -- (10, 3) -- (18, 3);
    	\draw (18, 7) -- (14.25, 7);
    \end{tikzpicture}

    
    
    \subsection{Cálculo de $Z_i$, $Z_o$, $A_i$ y $A_v$}
            
      \newcommand{\RB}{8447.20\,\Omega}
      \newcommand{\RC}{1.2\,\text{k}\Omega}
      \newcommand{\RL}{1\,\text{k}\Omega}
      \newcommand{\hfe}{484}
      \newcommand{\HIE}{1861.53\,\Omega} % poné 1942.21\,\Omega si usás I_{Cq,calc}

      Recordar que $R_B=R_1\parallel R_2$
      
      
      \noindent
      \begin{minipage}[t]{0.45\linewidth}
      \begin{align*}
      Z_i &= \frac{R_B \cdot h_{ie}}{R_B + h_{ie}} \\[6pt]
          &= \frac{\RB \cdot \HIE}{\RB + \HIE} \\[6pt]
          &= \boxed{1525.38\,\Omega} \\[16pt]
      Z_o &\simeq R_C = \boxed{1.2\,\text{k}\Omega}
      \end{align*}
      \end{minipage}\hfill
      \begin{minipage}[t]{0.45\linewidth}
      \begin{align*}
      A_i &= -h_{fe}\cdot \frac{R_C}{R_C+R_L}\cdot \frac{R_B}{R_B+h_{ie}} \\[6pt]
          &= -\hfe \cdot \frac{\RC}{\RC+\RL}\cdot \frac{\RB}{\RB+\HIE} \\[6pt]
          &= \boxed{-216.33} \\[16pt]
      A_v &= A_i \cdot \frac{R_L}{Z_i} \\[6pt]
          &= (-216.33)\cdot \frac{\RL}{1525.38\,\Omega} \\[6pt]
          &= \boxed{-141.82}
      \end{align*}
      \end{minipage}    

  \section{Análisis experimental}
    \begin{figure}[H]
      \centering
      \begin{minipage}[t]{0.35\textwidth}
        \centering
        \includegraphics[width=\linewidth]{pictures/pcb_completa_mediciones.jpg}
        \caption*{Montaje real}
      \end{minipage}\hfill
      \begin{minipage}[t]{0.59\textwidth}
        \centering
        \resizebox{\linewidth}{!}{%
          \begin{tikzpicture}
            \draw (7.25, 3) to[sinusoidal voltage source, l_={$V_{gen}\,(1\,\mathrm{kHz})$}, label distance=0.02cm] (7.25, 6);
            \draw (7.25, 6) to[american resistor, l={$R_s=2.2\,\mathrm{k}\Omega$}] (12, 6);
            \node[shape=rectangle, draw, line width=1pt, minimum width=3.215cm, minimum height=4.965cm](N1) at (13.625, 4.5){} node[anchor=center] at (N1.text){$CIRCUITO$};
            \draw (7.25, 3) -- (12, 3);
            \draw (8, 8) to[qvprobe, l={$V_s$}] (11, 8);
            \draw (10.75, 6) to[qvprobe, l={$V_i$}] (10.75, 3);
            \draw (8, 8) -| (8, 6);
            \draw (11, 8) -- (11, 6);
            \draw (18.75, 6) to[american resistor, l={$R_L=1\,\mathrm{k}\Omega$}] (18.75, 3);
            \draw (18.75, 6) -- (15.25, 6);
            \draw (18.75, 3) -- (15.25, 3);
            \draw (16, 6) to[qvprobe, l={$V_o=1\,V_{pp}$}] (16, 3);
          \end{tikzpicture}
        }
        \caption*{Esquema de medición}
      \end{minipage}
      \caption{Esquemático de medición experimenta y experiencia en laboratorio.}
      \label{fig:foto-tikz-inline}
    \end{figure}

    Para la medición de pequeña señal de forma experimental, se suelda una resistencia de $2\text{K}2\Omega$ a la entrada
    del circuito, la cual va a ser utilizada como Resistencia sensora. también se sueldan los correspondientes capacitores
    de acople y desacople.

    Luego se procede a ajustar la tensión de entrada de tal forma que a la salida se obtenga una onda senoidal con una
    tensión de $V_{pp}$ y se realizan las mediciones tanto de $V_s$ como de $V_i$ como se indica en el gráfico de
    arriba.

    \begin{figure}[!htbp]
      \centering
      \begin{minipage}[t]{0.32\textwidth}
        \centering
        \includegraphics[width=\linewidth]{pictures/osc_vi_0.5ms-20mv-sondax10-oscx5.jpg}
        \caption*{\small $V_i$ con escalas: 0.5ms/div, 20mv/div-Sonda X10-Osc X5.}
      \end{minipage}\hfill
      \begin{minipage}[t]{0.32\textwidth}
        \centering
        \includegraphics[width=\linewidth]{pictures/osc_vS_0.5ms-20mv-sondax10-oscx5.jpg}
        \caption*{\small $V_S$ con escalas: 0.5ms/div-10mv/div-Sonda X10-Osc X5.}
      \end{minipage}\hfill
      \begin{minipage}[t]{0.32\textwidth}
        \centering
        \includegraphics[width=\linewidth]{pictures/osc_vl_0.2ms-0.2v.jpg}
        \caption*{\small $V_L$ con escalas 0.20ms/div-200mv/div.}
      \end{minipage}
      \caption{Mediciones en R sensora, entrada y salida.}
    \end{figure}
    

    \vspace{0.5cm}
    A partir de las mediciones se tiene:
    \[
    V_S = 22\,\text{mV}, \qquad V_i = 13\,\text{mV}
    \]

    Por la Ley de Ohm:
    \[
    I_i = \frac{V_S - V_i}{R_S}
         = \frac{22\,\text{mV} - 13\,\text{mV}}{2.2\,\text{k}\Omega}
         = 4.091\,\mu\text{A}
    \]

    
    \noindent
    \begin{minipage}[t]{0.48\linewidth}
        \subsection*{Impedancia de entrada}
        \[
        Z_i = \frac{V_i}{I_i}
             = \frac{13\,\text{mV}}{4.091\,\mu\text{A}}
             = \boxed{3.177\,\text{k}\Omega}
        \]
    
        \subsection*{Ganancia de tensión $A_v$}
        \[
        A_v=\frac{V_O}{V_i}
             =\frac{1\,\text{V}_{pp}}{13\,\text{mV}}
             =\frac{1}{0.013}
             =\boxed{76.92}
        \]
    \end{minipage}\hfill
    \begin{minipage}[t]{0.48\linewidth}
        \subsection*{Ganancia de corriente $A_i$}
        \[
        A_i=\frac{i_O}{i_i}
            =\frac{\dfrac{V_O}{R_L}}{\dfrac{V_S - V_i}{R_{\text{sensor}}}}
        \]
        \[
        A_i=\frac{\dfrac{1\,\text{V}}{1\,\text{k}\Omega}}{\dfrac{22\,\text{mV}-13\,\text{mV}}{2.2\,\text{k}\Omega}}
            =\frac{\dfrac{1}{1000}}{\dfrac{0.022-0.013}{2200}}
        \]
        \[
        A_i=\frac{0.001}{\dfrac{0.009}{2200}}
            =\frac{0.001}{4.0909\times10^{-6}}
            =\boxed{244.4}
        \]
    \end{minipage}
    \vspace{0.5cm}

      Cabe destacar que tanto la ganancia de tensión como la de corriente aparecen con signo positivo. Esto difiere del 
      cálculo analítico, donde el análisis muestra la inversión de fase y por eso aparece el signo negativo. La razón 
      es que en la práctica no se está considerando un instante puntual de la señal, sino sus valores pico; por lo 
      tanto, no se puede reflejar si la fase está invertida o no.

    \subsection{Impedancia de salida}
    Para la medición de la impedancia de salida, se pone el generador a la salida y el mismo se configura para que con
    una resistencia sensora, haya $V_{PP}$ a la salida como se muestra en el siguiente gráfico
    \begin{figure}[H]
      \centering
    \begin{tikzpicture}
    	% Paths, nodes and wires:
    	\node[shape=rectangle, draw, line width=1pt, minimum width=3.215cm, minimum height=4.965cm](N1) at (13.625, 4.5){} node[anchor=center] at (N1.text){$CIRCUITO$};
    	\draw (10, 3) -- (12, 3);
    	\draw (15.25, 6) to[american resistor, l={$R_s = 1K \Omega$}] (20.75, 6);
    	\draw (20.75, 3) -- (15.25, 3);
    	\draw (16, 6) to[qvprobe, l={$V_o =1V_{PP}$}] (16, 3);
    	\draw (10, 3) |- (12, 6);
    	\draw (20.75, 6) to[sinusoidal voltage source, l={$\:1KHz$}] (20.75, 3);
    	\draw (16, 8) to[qvprobe, l={$V_s$}] (20, 8);
    	\draw (16, 8) -- (16, 6);
    	\draw (20, 8) -- (20, 6);
    \end{tikzpicture}
    \end{figure}

    \vspace{0.5cm}

    En nuestro caso, abrimos la salida y usamos la $R_L$ como $R_S$.

    La impedancia de salida se obtiene como:

    \[
    Z_o = \frac{V_o}{i_o} 
         = \frac{V_o}{\dfrac{V_s - V_o}{R_s}}
    \]
    
    Reemplazando con los valores medidos:
    
    \[
    Z_o = \frac{1\,\text{V}}{\dfrac{1.85\,\text{V} - 1\,\text{V}}{1\,\text{k}\Omega}}
    \]
    
    \[
    Z_o = \frac{1\,\text{V}}{\dfrac{0.85\,\text{V}}{1000\,\Omega}}
    \]
    
    \[
    Z_o = \frac{1}{0.00085}\,\Omega
    \]
    
    \[
    Z_o = \boxed{1176.47\,\Omega}
    \]

   

    

