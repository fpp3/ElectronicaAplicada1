\chapter{Introducción}

En este trabajo práctico de laboratorio se diseñó un amplificador con transistor JFET en configuración surtidor común, utilizando
auto-polarización bajo el criterio de máxima excursión simétrica (MES). A partir de los cálculos teóricos
se determinaron los valores de las resistencias de polarización y posteriormente se normalizaron a valores comerciales. En ambos
casos se realizaron simulaciones para comprobar que el punto de operación \emph{Q} no se desplazara fuera
de márgenes aceptables, y finalmente se efectuaron mediciones en el laboratorio con las resistencias normalizadas para comparar
los resultados con la simulación.

Además, se llevó a cabo un análisis de pequeña señal. Se calcularon de manera analítica la ganancia de tensión y corriente, así
como las impedancias de entrada y salida del amplificador. Luego, estos parámetros fueron medidos experimentalmente en el laboratorio
con el objetivo de contrastar los resultados teóricos con los prácticos.

Por último, se reemplazó la resistencia de drenador por una fuente de corriente basada en 2 transistores BJT P-N-P en
configuración espejo y se comprobaron sus beneficions.
